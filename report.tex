\documentclass[11pt,letterpaper]{article}
\usepackage{fullpage}
\usepackage[pdftex]{graphicx}
\usepackage{amsfonts,eucal,amsbsy,amsopn,amsmath}
\usepackage{url}
\usepackage[sort&compress]{natbib}
\usepackage{natbibspacing}
\usepackage{latexsym}
\usepackage{wasysym} 
\usepackage{rotating}
\usepackage{fancyhdr}
\DeclareMathOperator*{\argmax}{argmax}
\DeclareMathOperator*{\argmin}{argmin}
\usepackage{sectsty}
\usepackage[dvipsnames,usenames]{color}
\usepackage{multicol}
\definecolor{orange}{rgb}{1,0.5,0}
\usepackage{multirow}
\usepackage{sidecap}
\usepackage{caption}
\renewcommand{\captionfont}{\small}
\setlength{\oddsidemargin}{-0.04cm}
\setlength{\textwidth}{16.59cm}
\setlength{\topmargin}{-0.04cm}
\setlength{\headheight}{0in}
\setlength{\headsep}{0in}
\setlength{\textheight}{22.94cm}
\allsectionsfont{\normalsize}
\newcommand{\ignore}[1]{}
\newenvironment{enumeratesquish}{\begin{list}{\addtocounter{enumi}{1}\arabic{enumi}.}{\setlength{\itemsep}{-0.25em}\setlength{\leftmargin}{1em}\addtolength{\leftmargin}{\labelsep}}}{\end{list}}
\newenvironment{itemizesquish}{\begin{list}{\setcounter{enumi}{0}\labelitemi}{\setlength{\itemsep}{-0.25em}\setlength{\labelwidth}{0.5em}\setlength{\leftmargin}{\labelwidth}\addtolength{\leftmargin}{\labelsep}}}{\end{list}}

\bibpunct{(}{)}{;}{a}{,}{,}
\newcommand{\nascomment}[1]{\textcolor{blue}{\textbf{[#1 --NAS]}}}


\pagestyle{fancy}
\lhead{}
\chead{}
\rhead{}
\lfoot{}
\cfoot{\thepage~of \pageref{lastpage}}
\rfoot{}
\renewcommand{\headrulewidth}{0pt}
\renewcommand{\footrulewidth}{0pt}


\title{11-712:  NLP Lab Report}
\author{Bo Lei}
\date{April 26, 2013}

\begin{document}
\maketitle
\begin{abstract}
This paper is the report for CMU 11-712 NLP-lab. Bofinger is a French morphological analyzer that is going to be developed in this course.
\end{abstract}


\section{Basic Information about French}

Frech is a Romance language. It is one of the most popular language over the world and is majorly spoken by French people. The other areas where French is spoken as a First language are: Belgium, Quebec, Swizerland and Cote d'Ivore etc.\\
\\
There are many types of morphology in French words. For example, the morphology of French verbs can be divided into 3 groups:\\

\indent\indent The first group consists of the verbs that end with "er"\\
\indent\indent The second group consists of the verbs that end with "ir"\\
\indent\indent The third group contains irregular verbs.\\
\\
Each of the verb is subject to change according to the tense (present, past etc).\\
\\
French nouns have gender (masculin ou feminin) and number. This is indicated by the determinant. \\
\\
Adjactives and adverbs also change their number and sex according to the noun that they decorate.
\\
There are rules for each kind of inflection. However exceptions also exist in each case.


\section{Past Work on the Morphology of French}
Many works have been done on French Morphology. For instance, Xerox Morphological Analysis is a comercial NLP tool that supports the anayssis of multiple languages including French. It is a mature application and of high accuracy.\\
\\
The Lefff\footnote{http://www.labri.fr/perso/clement/lefff/} is a freely available and large-coverage morphological and syntactic lexicon for French developed by Benoˆıt Sagot. \\
\\
In addition, Morphalou\footnote{http://cnrtl.fr/lexiques/morphalou/} is also a Lexical Markup Framework based modle for morphological lexicon, developed by Centre National de Ressources Textuelles et Lexicales. 

\section{Available Resources}

\nascomment{include discussion of your corpora}

\section{Survey of Phenomena in French}

\section{Initial Design}

\section{System Analysis on Corpus A}

\section{Lessons Learned and Revised Design}

\section{System Analysis on Corpus B}

\section{Final Revisions}

\section{Future Work}





\bibliographystyle{plainnat}
\bibliography{refs}
\label{lastpage}
\end{document}
